% Template for the submission to:
%   The Annals of Probability           [aop]
%   The Annals of Applied Probability   [aap]
%   The Annals of Statistics            [aos]
%   The Annals of Applied Statistics    [aoas]
%   Stochastic Systems                  [ssy]
%
%Author: In this template, the places where you need to add information
%        (or delete line) are indicated by {???}.  Mostly the information
%        required is obvious, but some explanations are given in lines starting
%Author:
%All other lines should be ignored.  After editing, there should be
%no instances of ??? after this line.

% use option [preprint] to remove info line at bottom
% journal options: aop,aap,aos,aoas,ssy
% natbib option: authoryear
\documentclass[aoas,preprint,authoryear]{imsart}

\usepackage{amsthm,amsmath,natbib}
\RequirePackage[colorlinks,citecolor=blue,urlcolor=blue]{hyperref}

% provide arXiv number if available:
%\arxiv{arXiv:0000.0000}

% put your definitions there:
\startlocaldefs
\endlocaldefs

\begin{document}

\begin{frontmatter}
\title{Learning to Transfer through\\
       Generative Adversarial Neural Networks}

% indicate corresponding author with \corref{}
% \author{\fnms{John} \snm{Smith}\corref{}\ead[label=e1]{smith@foo.com}\thanksref{t1}}
% \thankstext{t1}{Thanks to somebody}
% \address{line 1\\ line 2\\ printead{e1}}
% \affiliation{Some University}

\author{\fnms{Gilles} \snm{Louppe}}
\affiliation{New York University}

\begin{abstract}

Lorem ipsum dolor sit amet, consectetur adipiscing elit. Nulla euismod feugiat
efficitur. Vivamus pulvinar imperdiet elit, et pharetra ipsum tincidunt ut. Cras
sollicitudin pharetra fringilla. Proin tempus ornare arcu, ac volutpat neque
efficitur eu. Ut et quam eleifend, facilisis sapien vitae, finibus dui. Nam
iaculis ante id metus venenatis, sit amet lobortis elit scelerisque. Quisque
imperdiet volutpat mi, non placerat quam auctor in. Fusce vel consequat ipsum.
In vestibulum quis est eu rutrum. Ut nec orci eu diam gravida rutrum id sit amet
dui. Donec nec est aliquet, pulvinar elit non, tristique ex. Donec eu fermentum
elit.

\end{abstract}

\end{frontmatter}

\section{Introduction}

Maecenas tempor nulla eget mi accumsan, non accumsan elit sagittis. Mauris quis
erat metus. Cras semper magna lectus, vitae interdum libero vehicula et. Proin
eget fermentum tortor. Pellentesque in auctor ipsum. Donec non mauris non tellus
vulputate vehicula. Nulla interdum odio bibendum consequat lobortis.

Mauris non volutpat mauris. Donec condimentum felis in tincidunt sollicitudin.
Pellentesque habitant morbi tristique senectus et netus et malesuada fames ac
turpis egestas. Nullam ut risus sed leo ornare volutpat. Morbi massa nisl,
bibendum nec enim ac, mollis placerat lorem. Aliquam eget mauris at libero
vehicula convallis vitae id urna. Donec at finibus tellus, quis iaculis nibh.
Vestibulum quis odio felis. Suspendisse finibus, magna at laoreet tincidunt,
urna nunc pellentesque risus, id pulvinar ipsum dolor sed elit. Sed sollicitudin
augue vel quam laoreet, eget egestas ipsum dignissim. Mauris commodo, dui ac
tempor facilisis, libero ipsum pellentesque mi, quis maximus sem libero nec
ante. Cras ac luctus odio. Phasellus pretium mi in fermentum pulvinar. Sed nec
lorem magna.

\section{Method}

\section{Experiments}

\section{Related work}

\section{Conclusions}

% Outline:
% - transfer learning has traditionally been solved through density ratio reweighting
% - works fine, but has the big issue of not working if the support of the source density does not cover the target density support
% - in this work, we propose instead a new paradigm for solving transfer learning
% - build a generative model for transforming samples from p1 to reproduce p0
% - adversarial learning from Goodfellow, with the practical difference that we often have only a finite set of samples (we need to avoid learning a lookup table)
% - regularization
% - network architecture (propagate x, similar to highway networks)
% - experiments
% - application: fix simulated data

\end{document}
